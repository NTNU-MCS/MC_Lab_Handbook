\addcontentsline{toc}{section}{Preface}
\section*{Preface}
The purpose of this document is to provide a manual that ease the process of using CyberShip Arctic Drillship(CSAD), and concerns only software and hardware of CSAD specifically. For information about the Marine Cybernetics Laboratory(MCLab) and how to implement custom control systems on the vessel, the reader is referred to the MCLab Handbook, which can be found on GitHub: \url{https://github.com/NTNU-MCS/MC_Lab_Handbook}

\addcontentsline{toc}{section}{Structure of document}
\section*{Structure of document}
This User Manual is divided in three parts: 
\begin{itemize}
	\item Technical description(hardware, software, mathematical models etc.)
	\item Operation manual(launching, operation and demolition instructions)
	\item In the Appendix, a description of the extended IMU system is given(4 IMUs)
\end{itemize}

\addcontentsline{toc}{section}{CSAD main data}
\begin{table*}[htb!]
	\centering
	\caption{CSAD main data}
	\begin{tabular}{ll}
		\hline
		\textbf{Parameter} & \textbf{Value} \\ \hline
		Length over all & 2.578 [m] \\
		Beam & 0.440 [m] \\
		Depth & 0.211 [m]\\
		Design draft & 0.133 [m]\\
		Weight & 127.92 [kg] \\
		Scale & 1:90\\
		IP-address(port 1) & 192.168.0.55 \\
		IP-address(port 2) & 192.168.1.21 \\
		RPi IP-address & 192.168.1.33 \\
		RPi Port Number & 51717 \\
		Qualisys body\footnotetext{Body-coordinate of highest marker} & (960, -190, -575) [mm]\\ 
		MATLAB Version & 2016b\\
		LabVIEW Version & 2017\\
		VeriStand Version & 2017\\
		\hline
	\end{tabular}
\end{table*}

\addcontentsline{toc}{section}{Lnown errors and further work}
\section*{Known errors and further work}
There are some known errors and weaknesses on CSAD: 
\begin{itemize}
	\item The screw between the gear and the servo controlling the angle of the thrusters have loosened previously(resulting in no control of the angle). It is suggested to add a lock washer(sprengskive) between the screw and the gear.
	\item There are new constraints on maximum thrust. From the towing test carried out in Juny 2017, a bollard pull test was performed. It is suggested to implement this new maximum thrust values in the thrust allocation. 
	\item The weight of the vessel is not correct, as it does not take the moon pool into account. 
\end{itemize}